\documentclass[a4paper]{article}
\usepackage[cm]{fullpage}
\usepackage[serbian]{babel}
\usepackage[utf8]{inputenc}
\usepackage[T2A]{fontenc}
\pagestyle{empty}
\begin{document}
\begin{center}
\textbf{УВИТ, септембар2 2016, теоријски део}  
\end{center}
Име и презиме: \hrulefill, Број индекса: \hrulefill, Наставник: \hrulefill
\begin{enumerate}

% 1
\item Шта је то мрежна картица?

\hrulefill

\hrulefill

\hrulefill

% 2
\item Укратко описати начин комуникације у мрежи са заједничким комуникационим каналом.

\hrulefill

\hrulefill

\hrulefill

% 3
\item Описати комутирање порука.

\hrulefill

\hrulefill

\hrulefill

% 4
\item Описати задатак мрежног слоја и задатак транспортног слоја у референтном вишеслојном моделу.

\hrulefill

\hrulefill

\hrulefill

% 5
\item Шта је то NAT?

\hrulefill

\hrulefill

\hrulefill

% 6
\item Укратко описати разлику између POP3 и IMAP протокола.

\hrulefill

\hrulefill

\hrulefill

% 7
\item Како се DTD код SGML-а наводи кроз унутрашњу декларацију?

\hrulefill

\hrulefill

\hrulefill

% 8
\item Направити у HTML-у табелу са испитним роковима и даумима полагања писменог и усменог дела испита из УВИТ-а. 
Једној ћелији са називом испитног рока треба да одговарају две ћелије - са датумом за писмени и са датумом за усмени део испита.


\hrulefill

\hrulefill

\hrulefill

% 9
\item Шта у језику CSS бива изабрано селектором \verb|h1.x, #m a.z|?

\hrulefill

\hrulefill

\hrulefill

% 10
\item Како се код CSS врши избор елемената код кога је вредност атрибута једнака задатој вредности?

\hrulefill

\hrulefill

\hrulefill

% 11
\item Дефинисати у језику JavaScript објекат \verb|ocene| који представља
  оцене студента, а чији су низ парова предмет-оцена, као и метод \verb|najmanja()| којом се 
	израчунава назив предмета из ког је студент добио најмању оцену.

\hrulefill

\hrulefill

\hrulefill

% 12
\item Шта је потребно написати да би се могле користити JQuery функције на HTML страни?

\hrulefill

\hrulefill

\hrulefill

% 13
\item Од којих се делова састоји URL?


\hrulefill

\hrulefill

\hrulefill

% 14
\item Дефинисати у језику PHP асоцијативни низ који садржи називе три испитна рока и
датуме писмених за УВИТ у та три рока, те написати код за
итеративни пролазак кроз тај низ и приказ података из низа.

\hrulefill

\hrulefill

\hrulefill

% 15
\item Навести пример у језику PHP како се преко супер-глобалних променљивих процесирају параметри који су пренесни методом GET.

\hrulefill

\hrulefill

% 16
\item Описати шта је то страни кључ код база података. Навести илустрацију, тј. пример.

\hrulefill

\hrulefill

\hrulefill

% 17
\item Написати упит којим се из табеле \verb|ocene(ocenaId, predmetSifra, studentIndeks, konacnaOcena)| 
пребројава колико студената има коју оцену (колико је петица, шестица, итд. десетки - 
посебно за сваки од предмета који се налазе у табели \verb|ocene|).

\hrulefill

\hrulefill

\hrulefill

% 18
\item Шта је све потребно претходно урадити да би функција \verb|mysqli_fetch_assoc| могла успешно да се примени?

\hrulefill

\hrulefill

\hrulefill

% 19
\item Како се реализује AJAX позив?

\hrulefill

\hrulefill

\hrulefill

% 20
\item Препоруке за коришћење слика како би се олакшао рад претраживачима?

\hrulefill

\hrulefill

\hrulefill
\end{enumerate}
\end{document}