\documentclass[a4paper]{article}
\usepackage[cm]{fullpage}
\usepackage[serbian]{babel}
\usepackage[utf8]{inputenc}
\usepackage[T2A]{fontenc}
\pagestyle{empty}

\begin{document}

\begin{center}
\textbf{УВИТ, јануар 2019, теоријски де испита, група А}  
\end{center}

Име и презиме: \hrulefill, Број индекса: \hrulefill

\begin{enumerate}

% 1
\item Описати који су основни задаци слоја везе података (data link layer). \hrulefill

\hrulefill

\hrulefill

\hrulefill

\hrulefill

% 2
\item Шта је то рутер, чему служи, које су му карактеристике, на ком слоју ради? \hrulefill

\hrulefill

\hrulefill

\hrulefill

\hrulefill

% 3
\item Шта је то адреса за јавно емитовање (brodcast address)? Како се одређује њена вредност?\hrulefill

\hrulefill

\hrulefill

\hrulefill

\hrulefill

% 4
\item Шта је то табела рутирања, где се налази и чему служи?\hrulefill


\hrulefill

\hrulefill

\hrulefill

\hrulefill

% 5
\item Описати разлике између метода GET и POST код HTTP протокола.\hrulefill

\hrulefill

\hrulefill

\hrulefill

\hrulefill

% 6
\item Који елементи на веб страни су изабрани селектором \verb|p.small img.desc|? \hrulefill

\hrulefill

\hrulefill


% 7 
\item Које све програмске парадигме подржава језик ЈаваСкрипт?\hrulefill

\hrulefill

% 8 
\item Шта је то Стек у оквиру ЈаваСкрипт машине, чему служи и како ради? \hrulefill

\hrulefill

\hrulefill

\hrulefill

\hrulefill

% 9
\item У чему је разлика између целих бројева и бројева у покретном зарезу у ЈаваСкрипту? \hrulefill

\hrulefill

\hrulefill 

\hrulefill

\hrulefill

% 10
\item Описати разлику између статичког (лексичког) и динамчког опсега дефинисаности. \hrulefill

\hrulefill

\hrulefill 

\hrulefill

\hrulefill

% 11
\item Описати процес дизања променљиве (hoisting)? \hrulefill

\hrulefill 

\hrulefill 

\hrulefill

\hrulefill

% 12
\item Навести бар четири вредности које ће имплицитном конверзијом бити претворене у \verb|false|? \hrulefill

\hrulefill 


% 13
\item Како се постиже да опсег важења бројачке променљиве \verb|for| циклуса буде шири од циклуса у ком се користи?  \hrulefill

\hrulefill

\hrulefill

\hrulefill

\hrulefill

% 14
\item Описати два начина за приступ особинама објекта. Навести пример. \hrulefill

\hrulefill

\hrulefill

\hrulefill

\hrulefill

\hrulefill

\hrulefill

\hrulefill

\hrulefill

\hrulefill

\hrulefill

% 15
\item Написати смислени пример ЈаваСкрипт кода код кога функција генерише другу функцију, па се потом позива новогенерисана функција. 

\hrulefill

\hrulefill

\hrulefill

\hrulefill

\hrulefill

\hrulefill

\hrulefill

\hrulefill

\hrulefill

\hrulefill


\end{enumerate}
\end{document}