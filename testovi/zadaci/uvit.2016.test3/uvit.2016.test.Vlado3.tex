\documentclass[a4paper]{article}
\usepackage[cm]{fullpage}
\usepackage[serbian]{babel}
\usepackage[utf8]{inputenc}
\usepackage[T2A]{fontenc}
\pagestyle{empty}
\begin{document}
\begin{center}
\textbf{УВИТ, теоријски тест}  
\end{center}
Име и презиме: \hrulefill, Број индекса: \hrulefill
\begin{enumerate}

% 1 
\item Како се све корисник може повезати на Интернет? 

\hrulefill

\hrulefill

\hrulefill

% 2 
\item Укратко описати шта је то LAN  и које су му карактеристике. 

\hrulefill

\hrulefill

\hrulefill

% 3 
\item Описати шта је то пакетно комутирање и где се користи.

\hrulefill

\hrulefill

\hrulefill

% 4
\item Када је настао Интернет? Када је настао веб? Описати укратко процес њиховог настанка.

\hrulefill

\hrulefill

\hrulefill

% 5
\item Чему служи протокол FTP? На ком је он нивоу? Укратко описати његове карактеристике.

\hrulefill

\hrulefill

\hrulefill

% 6
\item Шта је DHCP? Како он функционише?

\hrulefill

\hrulefill

\hrulefill

% 7 
\item По чему су специфичне адресе облика 192.168.х.х?  

\hrulefill

\hrulefill

\hrulefill

% 8
\item На које се све начине може исти стил придружити елементима на већем броју HTML страна?

\hrulefill

\hrulefill

\hrulefill

% 9
\item Шта у језику CSS бива изабрано селектором \verb|а.lepo, #silja p.c|?

\hrulefill

\hrulefill

\hrulefill

% 10
\item Како се извршавају клијентске скрипте?

\hrulefill

\hrulefill

\hrulefill


\end{enumerate}
\end{document}